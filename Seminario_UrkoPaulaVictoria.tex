% Options for packages loaded elsewhere
\PassOptionsToPackage{unicode}{hyperref}
\PassOptionsToPackage{hyphens}{url}
%
\documentclass[
]{article}
\usepackage{amsmath,amssymb}
\usepackage{iftex}
\ifPDFTeX
  \usepackage[T1]{fontenc}
  \usepackage[utf8]{inputenc}
  \usepackage{textcomp} % provide euro and other symbols
\else % if luatex or xetex
  \usepackage{unicode-math} % this also loads fontspec
  \defaultfontfeatures{Scale=MatchLowercase}
  \defaultfontfeatures[\rmfamily]{Ligatures=TeX,Scale=1}
\fi
\usepackage{lmodern}
\ifPDFTeX\else
  % xetex/luatex font selection
\fi
% Use upquote if available, for straight quotes in verbatim environments
\IfFileExists{upquote.sty}{\usepackage{upquote}}{}
\IfFileExists{microtype.sty}{% use microtype if available
  \usepackage[]{microtype}
  \UseMicrotypeSet[protrusion]{basicmath} % disable protrusion for tt fonts
}{}
\makeatletter
\@ifundefined{KOMAClassName}{% if non-KOMA class
  \IfFileExists{parskip.sty}{%
    \usepackage{parskip}
  }{% else
    \setlength{\parindent}{0pt}
    \setlength{\parskip}{6pt plus 2pt minus 1pt}}
}{% if KOMA class
  \KOMAoptions{parskip=half}}
\makeatother
\usepackage{xcolor}
\usepackage[margin=1in]{geometry}
\usepackage{color}
\usepackage{fancyvrb}
\newcommand{\VerbBar}{|}
\newcommand{\VERB}{\Verb[commandchars=\\\{\}]}
\DefineVerbatimEnvironment{Highlighting}{Verbatim}{commandchars=\\\{\}}
% Add ',fontsize=\small' for more characters per line
\usepackage{framed}
\definecolor{shadecolor}{RGB}{248,248,248}
\newenvironment{Shaded}{\begin{snugshade}}{\end{snugshade}}
\newcommand{\AlertTok}[1]{\textcolor[rgb]{0.94,0.16,0.16}{#1}}
\newcommand{\AnnotationTok}[1]{\textcolor[rgb]{0.56,0.35,0.01}{\textbf{\textit{#1}}}}
\newcommand{\AttributeTok}[1]{\textcolor[rgb]{0.13,0.29,0.53}{#1}}
\newcommand{\BaseNTok}[1]{\textcolor[rgb]{0.00,0.00,0.81}{#1}}
\newcommand{\BuiltInTok}[1]{#1}
\newcommand{\CharTok}[1]{\textcolor[rgb]{0.31,0.60,0.02}{#1}}
\newcommand{\CommentTok}[1]{\textcolor[rgb]{0.56,0.35,0.01}{\textit{#1}}}
\newcommand{\CommentVarTok}[1]{\textcolor[rgb]{0.56,0.35,0.01}{\textbf{\textit{#1}}}}
\newcommand{\ConstantTok}[1]{\textcolor[rgb]{0.56,0.35,0.01}{#1}}
\newcommand{\ControlFlowTok}[1]{\textcolor[rgb]{0.13,0.29,0.53}{\textbf{#1}}}
\newcommand{\DataTypeTok}[1]{\textcolor[rgb]{0.13,0.29,0.53}{#1}}
\newcommand{\DecValTok}[1]{\textcolor[rgb]{0.00,0.00,0.81}{#1}}
\newcommand{\DocumentationTok}[1]{\textcolor[rgb]{0.56,0.35,0.01}{\textbf{\textit{#1}}}}
\newcommand{\ErrorTok}[1]{\textcolor[rgb]{0.64,0.00,0.00}{\textbf{#1}}}
\newcommand{\ExtensionTok}[1]{#1}
\newcommand{\FloatTok}[1]{\textcolor[rgb]{0.00,0.00,0.81}{#1}}
\newcommand{\FunctionTok}[1]{\textcolor[rgb]{0.13,0.29,0.53}{\textbf{#1}}}
\newcommand{\ImportTok}[1]{#1}
\newcommand{\InformationTok}[1]{\textcolor[rgb]{0.56,0.35,0.01}{\textbf{\textit{#1}}}}
\newcommand{\KeywordTok}[1]{\textcolor[rgb]{0.13,0.29,0.53}{\textbf{#1}}}
\newcommand{\NormalTok}[1]{#1}
\newcommand{\OperatorTok}[1]{\textcolor[rgb]{0.81,0.36,0.00}{\textbf{#1}}}
\newcommand{\OtherTok}[1]{\textcolor[rgb]{0.56,0.35,0.01}{#1}}
\newcommand{\PreprocessorTok}[1]{\textcolor[rgb]{0.56,0.35,0.01}{\textit{#1}}}
\newcommand{\RegionMarkerTok}[1]{#1}
\newcommand{\SpecialCharTok}[1]{\textcolor[rgb]{0.81,0.36,0.00}{\textbf{#1}}}
\newcommand{\SpecialStringTok}[1]{\textcolor[rgb]{0.31,0.60,0.02}{#1}}
\newcommand{\StringTok}[1]{\textcolor[rgb]{0.31,0.60,0.02}{#1}}
\newcommand{\VariableTok}[1]{\textcolor[rgb]{0.00,0.00,0.00}{#1}}
\newcommand{\VerbatimStringTok}[1]{\textcolor[rgb]{0.31,0.60,0.02}{#1}}
\newcommand{\WarningTok}[1]{\textcolor[rgb]{0.56,0.35,0.01}{\textbf{\textit{#1}}}}
\usepackage{graphicx}
\makeatletter
\def\maxwidth{\ifdim\Gin@nat@width>\linewidth\linewidth\else\Gin@nat@width\fi}
\def\maxheight{\ifdim\Gin@nat@height>\textheight\textheight\else\Gin@nat@height\fi}
\makeatother
% Scale images if necessary, so that they will not overflow the page
% margins by default, and it is still possible to overwrite the defaults
% using explicit options in \includegraphics[width, height, ...]{}
\setkeys{Gin}{width=\maxwidth,height=\maxheight,keepaspectratio}
% Set default figure placement to htbp
\makeatletter
\def\fps@figure{htbp}
\makeatother
\setlength{\emergencystretch}{3em} % prevent overfull lines
\providecommand{\tightlist}{%
  \setlength{\itemsep}{0pt}\setlength{\parskip}{0pt}}
\setcounter{secnumdepth}{-\maxdimen} % remove section numbering
\ifLuaTeX
  \usepackage{selnolig}  % disable illegal ligatures
\fi
\IfFileExists{bookmark.sty}{\usepackage{bookmark}}{\usepackage{hyperref}}
\IfFileExists{xurl.sty}{\usepackage{xurl}}{} % add URL line breaks if available
\urlstyle{same}
\hypersetup{
  pdftitle={Seminario Fuentes de datos Biómédicas y Web Semántica},
  pdfauthor={Urko Alli Barrena, Paula Gregorio Losada, Victoria Garcia Lovelle},
  hidelinks,
  pdfcreator={LaTeX via pandoc}}

\title{Seminario Fuentes de datos Biómédicas y Web Semántica}
\usepackage{etoolbox}
\makeatletter
\providecommand{\subtitle}[1]{% add subtitle to \maketitle
  \apptocmd{\@title}{\par {\large #1 \par}}{}{}
}
\makeatother
\subtitle{Relación entre la Esperanza de Vida y el Agua}
\author{Urko Alli Barrena, Paula Gregorio Losada, Victoria Garcia
Lovelle}
\date{2023-12-06}

\begin{document}
\maketitle

{
\setcounter{tocdepth}{6}
\tableofcontents
}
\includegraphics{INPUT/escudo_ubu.jpg}

\hypertarget{introducciuxf3n}{%
\section{Introducción}\label{introducciuxf3n}}

En el desarrollo de este seminario, analizaremos cómo diversos aspectos
relacionados con la calidad y cantidad de agua consumida pueden influir
en la esperanza de vida en las diferentes Comunidades Autónomas de
España.

Exploraremos detalladamente cómo la calidad del agua, medida a través de
diferentes parámetros, y la cantidad de agua, pueden ser factores
determinantes para la salud y, consecuentemente, para la esperanza de
vida de las Comunidades Autónomas.

\hypertarget{objetivos}{%
\section{Objetivos}\label{objetivos}}

Este estudio tiene como objetivo examinar a fondo las conexiones entre
los factores medioambientales y la salud humana, centrándonos
específicamente en el papel crucial del agua como recurso esencial.

\begin{enumerate}
\def\labelenumi{\arabic{enumi}.}
\tightlist
\item
  Afectación de la calidad del agua en la Esperanza de Vida
\item
  Afectación de la cantidad de agua consumida en la Esperanza de Vida
\item
  Afectación del presupuesto para potabilizar agua con la cantidad de
  agua consumida
\item
  Impacto combinado de la calidad y cantidad de agua en la Esperanza de
  Vida
\end{enumerate}

\hypertarget{carga-de-libreruxedas}{%
\section{Carga de librerías}\label{carga-de-libreruxedas}}

\begin{enumerate}
\def\labelenumi{\arabic{enumi}.}
\tightlist
\item
  pdftools: Esta librería se utilizó para extraer datos de archivos PDF.
  En particular, se empleó para obtener información relevante sobre la
  calidad del agua de un informe en formato PDF.
\end{enumerate}

\begin{Shaded}
\begin{Highlighting}[]
\FunctionTok{library}\NormalTok{(pdftools)}
\end{Highlighting}
\end{Shaded}

\begin{enumerate}
\def\labelenumi{\arabic{enumi}.}
\setcounter{enumi}{1}
\tightlist
\item
  tidyverse: Un conjunto de paquetes que facilitan la manipulación y
  visualización de datos en R. Incluye librerías como tidyr y dplyr, las
  cuales fueron esenciales para organizar y transformar datos.
\end{enumerate}

\begin{Shaded}
\begin{Highlighting}[]
\FunctionTok{library}\NormalTok{(tidyverse)}
\end{Highlighting}
\end{Shaded}

\begin{enumerate}
\def\labelenumi{\arabic{enumi}.}
\setcounter{enumi}{2}
\tightlist
\item
  tidyjson: Esta librería fue útil para trabajar con datos en formato
  JSON. Fue esencial para analizar y extraer información de archivos
  JSON relacionados con la esperanza de vida y la cantidad de agua.
\end{enumerate}

\begin{Shaded}
\begin{Highlighting}[]
\FunctionTok{library}\NormalTok{(tidyjson)}
\end{Highlighting}
\end{Shaded}

\begin{enumerate}
\def\labelenumi{\arabic{enumi}.}
\setcounter{enumi}{3}
\tightlist
\item
  rjson: Se utilizó para cargar y procesar datos almacenados en formato
  JSON. Facilita la manipulación de datos estructurados y su conversión
  a formatos compatibles con R.
\end{enumerate}

\begin{Shaded}
\begin{Highlighting}[]
\FunctionTok{library}\NormalTok{(rjson) }
\end{Highlighting}
\end{Shaded}

\hypertarget{obtenciuxf3n-de-tablas}{%
\section{Obtención de tablas}\label{obtenciuxf3n-de-tablas}}

\hypertarget{tabla-de-esperanza-de-vida}{%
\subsubsection{1. Tabla de Esperanza de
Vida}\label{tabla-de-esperanza-de-vida}}

Para obtener la tabla de esperanza de vida se cargan los datos desde un
archivo Json.

Después organizan esos datos en un formato de tabla y realiza un
análisis para saber qué tipos contiene la variable EsperanzaVida.

\begin{Shaded}
\begin{Highlighting}[]
\NormalTok{archivoJson }\OtherTok{\textless{}{-}} \FunctionTok{fromJSON}\NormalTok{(}\AttributeTok{file =} \StringTok{"EsperanzaVida.json"}\NormalTok{)}

\NormalTok{esperanzaVida }\OtherTok{\textless{}{-}} \FunctionTok{spread\_all}\NormalTok{(archivoJson)}

\NormalTok{tibble1}\OtherTok{\textless{}{-}}\NormalTok{esperanzaVida }\SpecialCharTok{\%\textgreater{}\%} 
\NormalTok{  gather\_object }\SpecialCharTok{\%\textgreater{}\%}  
\NormalTok{  json\_types }\SpecialCharTok{\%\textgreater{}\%} 
  \FunctionTok{count}\NormalTok{(name, type)}
\end{Highlighting}
\end{Shaded}

Para obtener la tabla final se ha accedido a aquel que era de tipo array
que contiene los valores de interés y organiza estos datos en una tabla.

\begin{Shaded}
\begin{Highlighting}[]
\NormalTok{arrayData}\OtherTok{\textless{}{-}}\NormalTok{esperanzaVida }\SpecialCharTok{\%\textgreater{}\%}
  \FunctionTok{enter\_object}\NormalTok{(Data) }\SpecialCharTok{\%\textgreater{}\%} 
\NormalTok{  gather\_array }\SpecialCharTok{\%\textgreater{}\%} 
\NormalTok{  spread\_all }\SpecialCharTok{\%\textgreater{}\%} 
  \FunctionTok{select}\NormalTok{(Nombre, Anyo, Valor) }
\end{Highlighting}
\end{Shaded}

El atributo Nombre contiene una cadena de caracteres muy larga separada
por puntos. Entre toda la información estan los nombres de las
diferentes Comunidades Autónomas, por lo que se separa la cadena de
texto y se obtienen únicamente estos nombres.

\begin{Shaded}
\begin{Highlighting}[]
\NormalTok{partes }\OtherTok{\textless{}{-}} \FunctionTok{strsplit}\NormalTok{(arrayData}\SpecialCharTok{$}\NormalTok{Nombre, }\StringTok{"}\SpecialCharTok{\textbackslash{}\textbackslash{}}\StringTok{."}\NormalTok{)}

\NormalTok{comunidadesAutonomas}\OtherTok{\textless{}{-}}\FunctionTok{c}\NormalTok{()}
\ControlFlowTok{for}\NormalTok{ (i }\ControlFlowTok{in}\NormalTok{ partes)\{}
\NormalTok{  comunidadesAutonomas}\OtherTok{\textless{}{-}}\FunctionTok{c}\NormalTok{(comunidadesAutonomas,i[}\DecValTok{1}\NormalTok{])}
\NormalTok{\}}

\NormalTok{arrayData}\SpecialCharTok{$}\NormalTok{Nombre}\OtherTok{\textless{}{-}}\NormalTok{comunidadesAutonomas}
\end{Highlighting}
\end{Shaded}

De la tabla se obtiene el año 2020 de cada Comunidad Autónoma y se
obtiene como valor la media de la esperanza de vida en ese año.

Para tener todas las tablas con el mismo estilo, se han pasado los
nombres de las comunidades autónomas a mayúsculas.

\begin{Shaded}
\begin{Highlighting}[]
\NormalTok{tablaComunidadesAñoValor}\OtherTok{\textless{}{-}} \FunctionTok{as\_tibble}\NormalTok{(arrayData)}
\FunctionTok{attr}\NormalTok{(tablaComunidadesAñoValor, }\StringTok{"JSON"}\NormalTok{) }\OtherTok{\textless{}{-}} \ConstantTok{NULL}

\NormalTok{tablaEsperanzaDeVida }\OtherTok{\textless{}{-}}\NormalTok{ tablaComunidadesAñoValor }\SpecialCharTok{\%\textgreater{}\%}
  \FunctionTok{filter}\NormalTok{(Anyo}\SpecialCharTok{==}\DecValTok{2020}\NormalTok{)}\SpecialCharTok{\%\textgreater{}\%}
  \FunctionTok{group\_by}\NormalTok{(Anyo, Nombre) }\SpecialCharTok{\%\textgreater{}\%}
  \FunctionTok{summarize}\NormalTok{(}\AttributeTok{EsperanzaDeVida =} \FunctionTok{mean}\NormalTok{(Valor, }\AttributeTok{na.rm =} \ConstantTok{TRUE}\NormalTok{))}\SpecialCharTok{\%\textgreater{}\%}
  \FunctionTok{rename}\NormalTok{(Año}\OtherTok{=}\NormalTok{Anyo,}\AttributeTok{ComunidadAutonoma=}\NormalTok{Nombre)}

\NormalTok{tablaEsperanzaDeVidaFinal }\OtherTok{\textless{}{-}}\NormalTok{ tablaEsperanzaDeVida }\SpecialCharTok{\%\textgreater{}\%}
  \FunctionTok{mutate}\NormalTok{(}\AttributeTok{ComunidadAutonoma =} \FunctionTok{toupper}\NormalTok{(ComunidadAutonoma))}

\FunctionTok{colnames}\NormalTok{(tablaEsperanzaDeVidaFinal) }\OtherTok{\textless{}{-}} \FunctionTok{c}\NormalTok{(}\StringTok{"Anio"}\NormalTok{, }\StringTok{"ComunidadAutonoma"}\NormalTok{, }\StringTok{"EsperanzaDeVida"}\NormalTok{)}
\end{Highlighting}
\end{Shaded}

En la tabla final, se puede observar la media de la esperanza de vida de
cada Comunidad Autónoma en el año 2020.

\begin{Shaded}
\begin{Highlighting}[]
\NormalTok{tablaEsperanzaDeVidaFinal}
\end{Highlighting}
\end{Shaded}

\begin{verbatim}
## # A tibble: 19 x 3
## # Groups:   Anio [1]
##     Anio ComunidadAutonoma           EsperanzaDeVida
##    <dbl> <chr>                                 <dbl>
##  1  2020 ANDALUCÍA                              28.0
##  2  2020 ARAGÓN                                 28.4
##  3  2020 ASTURIAS, PRINCIPADO DE                28.1
##  4  2020 BALEARS, ILLES                         29.3
##  5  2020 CANARIAS                               29.1
##  6  2020 CANTABRIA                              28.9
##  7  2020 CASTILLA - LA MANCHA                   27.5
##  8  2020 CASTILLA Y LEÓN                        28.5
##  9  2020 CATALUÑA                               28.3
## 10  2020 CEUTA                                  26.7
## 11  2020 COMUNITAT VALENCIANA                   28.5
## 12  2020 EXTREMADURA                            27.9
## 13  2020 GALICIA                                29.3
## 14  2020 MADRID, COMUNIDAD DE                   28.0
## 15  2020 MELILLA                                26.3
## 16  2020 MURCIA, REGIÓN DE                      28.4
## 17  2020 NAVARRA, COMUNIDAD FORAL DE            29.0
## 18  2020 PAÍS VASCO                             28.8
## 19  2020 RIOJA, LA                              28.6
\end{verbatim}

\hypertarget{tabla-de-la-cantidad-de-agua}{%
\subsubsection{2. Tabla de la Cantidad de
Agua}\label{tabla-de-la-cantidad-de-agua}}

Para obtener la tabla de cantidad de agua se cargan los datos desde un
archivo Json.

Después organizan esos datos en un formato de tabla y realiza un
análisis para saber qué tipos contiene la variable EsperanzaVida.

\begin{Shaded}
\begin{Highlighting}[]
\NormalTok{archivoJsonCantidad }\OtherTok{\textless{}{-}} \FunctionTok{fromJSON}\NormalTok{(}\AttributeTok{file =} \StringTok{"CantidadAgua.json"}\NormalTok{)}

\NormalTok{cantidadAgua }\OtherTok{\textless{}{-}} \FunctionTok{spread\_all}\NormalTok{(archivoJsonCantidad)}

\NormalTok{cantidadAgua}\SpecialCharTok{\%\textgreater{}\%}
\NormalTok{  gather\_object}\SpecialCharTok{\%\textgreater{}\%}
\NormalTok{  json\_types}\SpecialCharTok{\%\textgreater{}\%}
  \FunctionTok{count}\NormalTok{(name,type)}
\end{Highlighting}
\end{Shaded}

\begin{verbatim}
## # A tibble: 3 x 3
##   name     type       n
##   <chr>    <fct>  <int>
## 1 Data     array    133
## 2 MetaData array    133
## 3 Nombre   string   133
\end{verbatim}

Para poder obtener la tabla final de la cantidad de agua consumida en
cada Comunidad autónoma, se ha accedido al atributo de tipo array que
contiene la información de interés que es Data y organiza estos datos en
una tabla.

Seguidamente, como únicamente necesitamos las comunidades autónomas y
están escritas en una larga cadena de texto, dividimos la cadena de
texto y obtenemos de ella el nombre de las Comunidades Autónomas.

Esa nueva variable con las Comunidades Autónomas se añaden a la tabla
final y se elimina donde ponga España.

\begin{Shaded}
\begin{Highlighting}[]
\NormalTok{arrayDataCantidad}\OtherTok{\textless{}{-}}\NormalTok{cantidadAgua}\SpecialCharTok{\%\textgreater{}\%}
  \FunctionTok{enter\_object}\NormalTok{(Data)}\SpecialCharTok{\%\textgreater{}\%}
\NormalTok{  gather\_array}\SpecialCharTok{\%\textgreater{}\%}
\NormalTok{  spread\_all}\SpecialCharTok{\%\textgreater{}\%}
  \FunctionTok{select}\NormalTok{(}\SpecialCharTok{{-}}\NormalTok{document.id,}\SpecialCharTok{{-}}\NormalTok{array.index)}

\NormalTok{cadenas }\OtherTok{\textless{}{-}} \FunctionTok{strsplit}\NormalTok{(arrayDataCantidad}\SpecialCharTok{$}\NormalTok{Nombre, }\StringTok{"}\SpecialCharTok{\textbackslash{}\textbackslash{}}\StringTok{,"}\NormalTok{)}

\NormalTok{comunidadesAutonomasCantidad}\OtherTok{\textless{}{-}}\FunctionTok{c}\NormalTok{()}
\ControlFlowTok{for}\NormalTok{ (i }\ControlFlowTok{in}\NormalTok{ cadenas)\{}
\NormalTok{  comunidadesAutonomasCantidad}\OtherTok{\textless{}{-}}\FunctionTok{c}\NormalTok{(comunidadesAutonomasCantidad,i[}\DecValTok{1}\NormalTok{])}
\NormalTok{\}}

\NormalTok{arrayDataCantidad}\SpecialCharTok{$}\NormalTok{Nombre}\OtherTok{\textless{}{-}}\NormalTok{comunidadesAutonomasCantidad}

\NormalTok{tabla }\OtherTok{\textless{}{-}}\NormalTok{ arrayDataCantidad }\SpecialCharTok{\%\textgreater{}\%}
  \FunctionTok{filter}\NormalTok{(}\SpecialCharTok{!}\NormalTok{(Nombre }\SpecialCharTok{==} \StringTok{"España"}\NormalTok{)) }\SpecialCharTok{\%\textgreater{}\%}
  \FunctionTok{select}\NormalTok{(Nombre, NombrePeriodo, Valor)}
\end{Highlighting}
\end{Shaded}

Se filtra únicamente por el año 2020, que es el año de interés, se
realiza una media de los valores obtenidos en el año 2020 de cada
comunicad autónoma.

Para tener todas las tablas con el mismo estilo, se han pasado los
nombres de las Comunidades Autónomas a mayúsculas y se han renombrado.

Por último se ha pasado a tipo entero la columna años.

\begin{Shaded}
\begin{Highlighting}[]
\NormalTok{tablaCantidadDeAgua }\OtherTok{\textless{}{-}}\NormalTok{ tabla }\SpecialCharTok{\%\textgreater{}\%}
  \FunctionTok{filter}\NormalTok{(NombrePeriodo  }\SpecialCharTok{==}\DecValTok{2020}\NormalTok{)}\SpecialCharTok{\%\textgreater{}\%}
  \FunctionTok{group\_by}\NormalTok{(NombrePeriodo  , Nombre) }\SpecialCharTok{\%\textgreater{}\%}
  \FunctionTok{summarize}\NormalTok{(}\AttributeTok{Cantidad =} \FunctionTok{mean}\NormalTok{(Valor, }\AttributeTok{na.rm =} \ConstantTok{TRUE}\NormalTok{))}\SpecialCharTok{\%\textgreater{}\%}
  \FunctionTok{rename}\NormalTok{(Año}\OtherTok{=}\NormalTok{NombrePeriodo,}\AttributeTok{ComunidadAutonoma=}\NormalTok{Nombre)}

\NormalTok{tablaCantidadDeAgua }\OtherTok{\textless{}{-}}\NormalTok{ tablaCantidadDeAgua }\SpecialCharTok{\%\textgreater{}\%}
  \FunctionTok{mutate}\NormalTok{(}\AttributeTok{ComunidadAutonoma =} \FunctionTok{toupper}\NormalTok{(ComunidadAutonoma))}

\NormalTok{tablaCantidadDeAgua[}\DecValTok{4}\NormalTok{,}\DecValTok{2}\NormalTok{] }\OtherTok{\textless{}{-}} \FunctionTok{c}\NormalTok{(}\StringTok{"BALEARS, ILLES"}\NormalTok{)}
\NormalTok{tablaCantidadDeAgua[}\DecValTok{8}\NormalTok{,}\DecValTok{2}\NormalTok{] }\OtherTok{\textless{}{-}} \FunctionTok{c}\NormalTok{(}\StringTok{"CASTILLA {-} LA MANCHA"}\NormalTok{)}
\NormalTok{tablaCantidadDeAgua[}\DecValTok{14}\NormalTok{,}\DecValTok{2}\NormalTok{] }\OtherTok{\textless{}{-}} \FunctionTok{c}\NormalTok{(}\StringTok{"MADRID, COMUNIDAD DE"}\NormalTok{)}
\NormalTok{tablaCantidadDeAgua[}\DecValTok{15}\NormalTok{,}\DecValTok{2}\NormalTok{] }\OtherTok{\textless{}{-}} \FunctionTok{c}\NormalTok{(}\StringTok{"MURCIA, REGIÓN DE"}\NormalTok{)}
\NormalTok{tablaCantidadDeAgua[}\DecValTok{16}\NormalTok{,}\DecValTok{2}\NormalTok{] }\OtherTok{\textless{}{-}} \FunctionTok{c}\NormalTok{(}\StringTok{"NAVARRA, COMUNIDAD FORAL DE"}\NormalTok{)}
\NormalTok{tablaCantidadDeAgua[}\DecValTok{18}\NormalTok{,}\DecValTok{2}\NormalTok{] }\OtherTok{\textless{}{-}} \FunctionTok{c}\NormalTok{(}\StringTok{"RIOJA, LA"}\NormalTok{)}

\FunctionTok{colnames}\NormalTok{(tablaCantidadDeAgua) }\OtherTok{\textless{}{-}} \FunctionTok{c}\NormalTok{(}\StringTok{"Anio"}\NormalTok{, }\StringTok{"ComunidadAutonoma"}\NormalTok{, }\StringTok{"Cantidad"}\NormalTok{)}

\NormalTok{tablaCantidadDeAguaFinal }\OtherTok{\textless{}{-}}\NormalTok{ tablaCantidadDeAgua }\SpecialCharTok{\%\textgreater{}\%}
  \FunctionTok{mutate\_at}\NormalTok{(}\FunctionTok{vars}\NormalTok{(Anio), as.integer)}
\end{Highlighting}
\end{Shaded}

En la tabla final de la cantidad de agua, se puede observar la cantidad
de agua consumida por cada Comunidad Autónoma en el año 2020.

\begin{Shaded}
\begin{Highlighting}[]
\NormalTok{tablaCantidadDeAguaFinal}
\end{Highlighting}
\end{Shaded}

\begin{verbatim}
## # A tibble: 18 x 3
## # Groups:   Anio [1]
##     Anio ComunidadAutonoma           Cantidad
##    <int> <chr>                          <dbl>
##  1  2020 ANDALUCÍA                    262867 
##  2  2020 ARAGÓN                        38689.
##  3  2020 ASTURIAS                      28342.
##  4  2020 BALEARS, ILLES                42033.
##  5  2020 CANARIAS                      79923.
##  6  2020 CANTABRIA                     20367.
##  7  2020 CASTILLA Y LEÓN               73272.
##  8  2020 CASTILLA - LA MANCHA          60920.
##  9  2020 CATALUÑA                     239767.
## 10  2020 CEUTA Y MELILLA                5507.
## 11  2020 COMUNITAT VALENCIANA         188854.
## 12  2020 EXTREMADURA                   28064.
## 13  2020 GALICIA                       71149 
## 14  2020 MADRID, COMUNIDAD DE         216232.
## 15  2020 MURCIA, REGIÓN DE             57000.
## 16  2020 NAVARRA, COMUNIDAD FORAL DE   22551.
## 17  2020 PAÍS VASCO                    52635.
## 18  2020 RIOJA, LA                      9448.
\end{verbatim}

\hypertarget{tabla-de-la-calidad-del-agua}{%
\subsubsection{3. Tabla de la Calidad del
Agua}\label{tabla-de-la-calidad-del-agua}}

Este código extrae y procesa información relevante sobre la calidad del
agua de un informe en PDF.

Primero, convierte el contenido del PDF a texto y luego selecciona un
rango específico de líneas en una página particular del documento, con
el objetivo de obtener datos clave relacionados con la calidad del agua.

\begin{Shaded}
\begin{Highlighting}[]
\NormalTok{ruta\_pdf }\OtherTok{\textless{}{-}} \FunctionTok{pdf\_text}\NormalTok{(}\StringTok{"report\_Cap.3\_part2.\_Libro\_blanco\_del\_agua.pdf"}\NormalTok{)}
\NormalTok{pagina }\OtherTok{\textless{}{-}}\NormalTok{ ruta\_pdf[}\DecValTok{17}\NormalTok{]}
\NormalTok{lineas }\OtherTok{\textless{}{-}} \FunctionTok{strsplit}\NormalTok{(pagina, }\StringTok{"}\SpecialCharTok{\textbackslash{}n}\StringTok{"}\NormalTok{)[[}\DecValTok{1}\NormalTok{]]}
\NormalTok{linea\_deseada }\OtherTok{\textless{}{-}}\NormalTok{ lineas[}\DecValTok{5}\SpecialCharTok{:}\DecValTok{21}\NormalTok{] }
\end{Highlighting}
\end{Shaded}

En este código se dividen las líneas deseadas usando espacios como
delimitadores. Luego, crea dos conjuntos de datos: datos\_obtenidos,
excluyendo las primeras tres dimensiones no necesarias, y
datosDeInteres, que contiene la información esencial sobre las
comunidades autónomas y los valores de la calidad del agua.

Además, se añaden manualmente filas específicas que no se extrajeron
correctamente del informe.

\begin{Shaded}
\begin{Highlighting}[]
\NormalTok{datos\_divididos }\OtherTok{\textless{}{-}} \FunctionTok{strsplit}\NormalTok{(linea\_deseada, }\StringTok{"}\SpecialCharTok{\textbackslash{}\textbackslash{}}\StringTok{s+"}\NormalTok{)}
\NormalTok{datos\_obtenidos }\OtherTok{\textless{}{-}}\NormalTok{ datos\_divididos[}\SpecialCharTok{{-}}\FunctionTok{c}\NormalTok{(}\DecValTok{1}\SpecialCharTok{:}\DecValTok{3}\NormalTok{)]}

\NormalTok{datosDeInteres }\OtherTok{\textless{}{-}}\NormalTok{ datos\_obtenidos[}\SpecialCharTok{{-}}\FunctionTok{c}\NormalTok{(}\FunctionTok{length}\NormalTok{(datos\_obtenidos))]}
\NormalTok{datosDeInteres[[}\DecValTok{4}\NormalTok{]] }\OtherTok{\textless{}{-}} \FunctionTok{c}\NormalTok{(}\StringTok{"CASTILLA {-} LA MANCHA"}\NormalTok{,}\StringTok{"28"}\NormalTok{,}\StringTok{"39"}\NormalTok{,}\StringTok{"43"}\NormalTok{,}\StringTok{"24"}\NormalTok{,}\StringTok{"7"}\NormalTok{,}\StringTok{"12"}\NormalTok{,}\StringTok{"0"}\NormalTok{)}
\NormalTok{datosDeInteres[[}\DecValTok{5}\NormalTok{]] }\OtherTok{\textless{}{-}} \FunctionTok{c}\NormalTok{(}\StringTok{"CASTILLA Y LEÓN"}\NormalTok{,}\StringTok{"2"}\NormalTok{,}\StringTok{"2"}\NormalTok{,}\StringTok{"2"}\NormalTok{,}\StringTok{"0"}\NormalTok{,}\StringTok{"1"}\NormalTok{,}\StringTok{"1"}\NormalTok{,}\StringTok{"0"}\NormalTok{)}
\NormalTok{datosDeInteres[[}\DecValTok{9}\NormalTok{]] }\OtherTok{\textless{}{-}} \FunctionTok{c}\NormalTok{(}\StringTok{"MADRID, COMUNIDAD DE"}\NormalTok{,}\StringTok{"6"}\NormalTok{,}\StringTok{"6"}\NormalTok{,}\StringTok{"7"}\NormalTok{,}\StringTok{"0"}\NormalTok{,}\StringTok{"2"}\NormalTok{,}\StringTok{"5"}\NormalTok{,}\StringTok{"0"}\NormalTok{)}
\NormalTok{datosDeInteres[[}\DecValTok{10}\NormalTok{]] }\OtherTok{\textless{}{-}} \FunctionTok{c}\NormalTok{(}\StringTok{"MURCIA, REGIÓN DE"}\NormalTok{,}\StringTok{"3"}\NormalTok{,}\StringTok{"3"}\NormalTok{,}\StringTok{"3"}\NormalTok{,}\StringTok{"0"}\NormalTok{,}\StringTok{"0"}\NormalTok{,}\StringTok{"3"}\NormalTok{,}\StringTok{"0"}\NormalTok{)}
\NormalTok{datosDeInteres[[}\DecValTok{11}\NormalTok{]] }\OtherTok{\textless{}{-}} \FunctionTok{c}\NormalTok{(}\StringTok{"NAVARRA, COMUNIDAD FORAL DE"}\NormalTok{,}\StringTok{"11"}\NormalTok{,}\StringTok{"11"}\NormalTok{,}\StringTok{"11"}\NormalTok{,}\StringTok{"4"}\NormalTok{,}\StringTok{"5"}\NormalTok{,}\StringTok{"2"}\NormalTok{,}\StringTok{"0"}\NormalTok{)}
\NormalTok{datosDeInteres[[}\DecValTok{12}\NormalTok{]] }\OtherTok{\textless{}{-}} \FunctionTok{c}\NormalTok{(}\StringTok{"RIOJA, LA"}\NormalTok{,}\StringTok{"1"}\NormalTok{,}\StringTok{"1"}\NormalTok{,}\StringTok{"1"}\NormalTok{,}\StringTok{"0"}\NormalTok{,}\StringTok{"1"}\NormalTok{,}\StringTok{"0"}\NormalTok{,}\StringTok{"0"}\NormalTok{)}
\NormalTok{datosDeInteres[[}\DecValTok{13}\NormalTok{]] }\OtherTok{\textless{}{-}} \FunctionTok{c}\NormalTok{(}\StringTok{"COMUNITAT VALENCIANA"}\NormalTok{,}\StringTok{"2"}\NormalTok{,}\StringTok{"2"}\NormalTok{,}\StringTok{"2"}\NormalTok{,}\StringTok{"0"}\NormalTok{,}\StringTok{"1"}\NormalTok{,}\StringTok{"1"}\NormalTok{,}\StringTok{"0"}\NormalTok{)}
\end{Highlighting}
\end{Shaded}

En cada iteración del bucle, se agrega la fila representada por el
elemento i al data frame tablaCalidadDeAgua utilizando la función
rbind().

Por último se renombra y se cambia a tipo numérico a aquellas columnas
que sea necesario.

\begin{Shaded}
\begin{Highlighting}[]
\NormalTok{tablaCalidadDeAgua}\OtherTok{\textless{}{-}}\FunctionTok{data.frame}\NormalTok{()}
\ControlFlowTok{for}\NormalTok{ (i }\ControlFlowTok{in}\NormalTok{ datosDeInteres)\{}
\NormalTok{  tablaCalidadDeAgua}\OtherTok{\textless{}{-}}\FunctionTok{rbind}\NormalTok{(tablaCalidadDeAgua,i)}
\NormalTok{\}}

\FunctionTok{colnames}\NormalTok{(tablaCalidadDeAgua) }\OtherTok{\textless{}{-}} \FunctionTok{c}\NormalTok{(}\StringTok{"ComunidadAutonoma"}\NormalTok{, }\StringTok{"NumdeMunicipios"}\NormalTok{, }\StringTok{"ZonasdeBaño"}\NormalTok{,}\StringTok{"PuntosdeMuestreo"}\NormalTok{,}\StringTok{"Aguas2"}\NormalTok{, }\StringTok{"Aguas1"}\NormalTok{,}\StringTok{"Aguas0"}\NormalTok{, }\StringTok{"AguasSCF"}\NormalTok{)}

\NormalTok{colNumericas }\OtherTok{\textless{}{-}} \FunctionTok{c}\NormalTok{(}\StringTok{"NumdeMunicipios"}\NormalTok{, }\StringTok{"ZonasdeBaño"}\NormalTok{,}\StringTok{"PuntosdeMuestreo"}\NormalTok{,}\StringTok{"Aguas2"}\NormalTok{, }\StringTok{"Aguas1"}\NormalTok{,}\StringTok{"Aguas0"}\NormalTok{, }\StringTok{"AguasSCF"}\NormalTok{)}
\NormalTok{tablaCalidadDeAguaFinal }\OtherTok{\textless{}{-}}\NormalTok{ tablaCalidadDeAgua }\SpecialCharTok{\%\textgreater{}\%}
  \FunctionTok{mutate\_at}\NormalTok{(}\FunctionTok{vars}\NormalTok{(colNumericas), as.integer)}
\end{Highlighting}
\end{Shaded}

\begin{Shaded}
\begin{Highlighting}[]
\NormalTok{tablaCalidadDeAguaFinal}
\end{Highlighting}
\end{Shaded}

\begin{verbatim}
##              ComunidadAutonoma NumdeMunicipios ZonasdeBaño PuntosdeMuestreo
## 1                    ANDALUCÍA              58          63               70
## 2                       ARAGÓN              11          11               12
## 3                     ASTURIAS               1           1                1
## 4         CASTILLA - LA MANCHA              28          39               43
## 5              CASTILLA Y LEÓN               2           2                2
## 6                     CATALUÑA               9          10               11
## 7                  EXTREMADURA              17          17               17
## 8                      GALICIA              53          54               68
## 9         MADRID, COMUNIDAD DE               6           6                7
## 10           MURCIA, REGIÓN DE               3           3                3
## 11 NAVARRA, COMUNIDAD FORAL DE              11          11               11
## 12                   RIOJA, LA               1           1                1
## 13        COMUNITAT VALENCIANA               2           2                2
##    Aguas2 Aguas1 Aguas0 AguasSCF
## 1       3     36     27        4
## 2       3      7      1        1
## 3       0      0      1        0
## 4      24      7     12        0
## 5       0      1      1        0
## 6       3      8      0        0
## 7       0      0      0       17
## 8      10     45     13        0
## 9       0      2      5        0
## 10      0      0      3        0
## 11      4      5      2        0
## 12      0      1      0        0
## 13      0      1      1        0
\end{verbatim}

\hypertarget{tabla-de-presupuesto-para-la-potabilizaciuxf3n-del-agua}{%
\subsubsection{4. Tabla de Presupuesto para la Potabilización del
Agua}\label{tabla-de-presupuesto-para-la-potabilizaciuxf3n-del-agua}}

Para obtener esta tabla se ha importado un archivo CSV utilizando la
función read\_csv y después se renombran las columnas para tener
unificados todos los nombres de las diferentes tablas.

\begin{Shaded}
\begin{Highlighting}[]
\NormalTok{summodificado }\OtherTok{\textless{}{-}} \FunctionTok{read\_csv}\NormalTok{(}\StringTok{"summodificado.csv"}\NormalTok{)}
\FunctionTok{colnames}\NormalTok{(summodificado) }\OtherTok{\textless{}{-}} \FunctionTok{c}\NormalTok{(}\StringTok{"TotalNacional"}\NormalTok{, }\StringTok{"ComunidadAutonoma"}\NormalTok{, }\StringTok{"GruposDeUsuarioEImporte"}\NormalTok{,}\StringTok{"Anio"}\NormalTok{, }\StringTok{"Presupuesto"}\NormalTok{)}
\end{Highlighting}
\end{Shaded}

Se realizan una serie de modificaciones para poder obtener la tabla
deseada.

Se seleccionan las filas donde la columna ``GruposDeUsuarioEImporte'' es
igual a ``Importe total de la inversión en los servicios de suministro''
y la columna ``Anio'' es igual a ``2020''.

Delante de cada nombre de la Comunidad Autónoma existe un número por lo
que se elimina y se convierten los nombres de las comunidades a
mayúsculas.

Se eliminan los puntos de la columna presupuesto y se pasa a tipo
entero. Y por último se descarta lo que no es necesario.

\begin{Shaded}
\begin{Highlighting}[]
\NormalTok{tablaPresupuestos }\OtherTok{\textless{}{-}}\NormalTok{ summodificado}\SpecialCharTok{\%\textgreater{}\%}
  \FunctionTok{filter}\NormalTok{(GruposDeUsuarioEImporte}\SpecialCharTok{==}\StringTok{"Importe total de la inversión en los servicios de suministro"} \SpecialCharTok{\&}\NormalTok{ Anio}\SpecialCharTok{==} \StringTok{"2020"}\NormalTok{) }\SpecialCharTok{\%\textgreater{}\%}
  \FunctionTok{mutate}\NormalTok{(}\AttributeTok{ComunidadAutonoma =} \FunctionTok{gsub}\NormalTok{(}\StringTok{"\^{}}\SpecialCharTok{\textbackslash{}\textbackslash{}}\StringTok{d+}\SpecialCharTok{\textbackslash{}\textbackslash{}}\StringTok{s*"}\NormalTok{, }\StringTok{""}\NormalTok{, ComunidadAutonoma)) }\SpecialCharTok{\%\textgreater{}\%}
  \FunctionTok{mutate}\NormalTok{(}\AttributeTok{ComunidadAutonoma =} \FunctionTok{toupper}\NormalTok{(ComunidadAutonoma)) }\SpecialCharTok{\%\textgreater{}\%}
  \FunctionTok{select}\NormalTok{ (}\AttributeTok{.data =}\NormalTok{ ., Anio, ComunidadAutonoma}\SpecialCharTok{:}\NormalTok{Presupuesto) }\SpecialCharTok{\%\textgreater{}\%}
  \FunctionTok{drop\_na}\NormalTok{()}

\NormalTok{tablaPresupuestos}\SpecialCharTok{$}\NormalTok{Presupuesto }\OtherTok{\textless{}{-}} \FunctionTok{gsub}\NormalTok{(}\StringTok{"}\SpecialCharTok{\textbackslash{}\textbackslash{}}\StringTok{."}\NormalTok{, }\StringTok{""}\NormalTok{, tablaPresupuestos}\SpecialCharTok{$}\NormalTok{Presupuesto)}
\NormalTok{tablaPresupuestos}\SpecialCharTok{$}\NormalTok{Presupuesto }\OtherTok{\textless{}{-}} \FunctionTok{as.integer}\NormalTok{(tablaPresupuestos}\SpecialCharTok{$}\NormalTok{Presupuesto)}
\NormalTok{tablaPresupuestosFinal }\OtherTok{\textless{}{-}}\NormalTok{ tablaPresupuestos[,}\SpecialCharTok{{-}}\DecValTok{3}\NormalTok{]}
\end{Highlighting}
\end{Shaded}

Asi obtenemos los datos de los presupuesto para la potabilización del
agua en el año 2020 en las diferentes Comunidades Autónomas de España

\begin{Shaded}
\begin{Highlighting}[]
\NormalTok{tablaPresupuestosFinal}
\end{Highlighting}
\end{Shaded}

\begin{verbatim}
## # A tibble: 17 x 3
##     Anio ComunidadAutonoma           Presupuesto
##    <dbl> <chr>                             <int>
##  1  2020 ANDALUCÍA                         43361
##  2  2020 ARAGÓN                              691
##  3  2020 ASTURIAS, PRINCIPADO DE             757
##  4  2020 BALEARS, ILLES                    14514
##  5  2020 CANARIAS                           4015
##  6  2020 CANTABRIA                           223
##  7  2020 CASTILLA Y LEÓN                    8083
##  8  2020 CASTILLA - LA MANCHA               2397
##  9  2020 CATALUÑA                          30222
## 10  2020 COMUNITAT VALENCIANA              39274
## 11  2020 EXTREMADURA                        2263
## 12  2020 GALICIA                            2594
## 13  2020 MADRID, COMUNIDAD DE              80532
## 14  2020 MURCIA, REGIÓN DE                  3170
## 15  2020 NAVARRA, COMUNIDAD FORAL DE        5704
## 16  2020 PAÍS VASCO                         3001
## 17  2020 RIOJA, LA                           304
\end{verbatim}

\end{document}
